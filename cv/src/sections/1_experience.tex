\sectionTitle{Experience}{}

\begin{experiences}

 \researchexperience
    {Present} {Sadie Bartholomew Algorithm Solutions}{Reading, UK}{}
    {Feb. 2024} {\textit{Private contractor}\\
    Conducting ad-hoc contract research software engineering work, such as for the University of Edinburgh.
    }
    
\emptySeparator
\researchexperience
    {Present} {National Centre for Atmospheric Science \& University of Reading | Dept. Meteorology}{Reading, UK}{}
    {Jan. 2020} {\textit{Senior Research Software Engineer \lbrack March 2025+ following promotion\rbrack, Computational Scientist \lbrack until Feb 2025\rbrack}\\
    On the core \href{https://cms.ncas.ac.uk/index.html}{NCAS-CMS (NCAS Computational Modelling Services)} team, providing software engineering support and delivering key underpinning infrastructure for the UK earth-system science community. Individual focus has varied by tenure but has generally been towards one of three overarching goals:
    \begin{itemize}
        \itemsep0em 
        \item developing, maintaining, optimising and consulting on (user support, etc.) open-source tooling related to the \href{https://cfconventions.org/}{CF Conventions metadata standard}, primarily \href{https://github.com/NCAS-CMS/cf-python}{\texttt{cf-python}}, \href{https://github.com/NCAS-CMS/cfdm}{\texttt{cfdm}} and \href{https://github.com/NCAS-CMS/cf-plot}{\texttt{cf-plot}}, as well as furthering and advocating for the standard such as providing information management, support and training;
        \item contributing to software products useful to the community making targeted use of the above libraries;
        \item delivering documentation infrastructure for model inter-comparison, notably \href{https://wcrp-cmip.org/cmip6/}{CMIP6} and beyond.
    \end{itemize}
    Indicators of role-based performance include being: moved directly from a fixed-term to a permanent contract;  awarded an additional spinal point increment in 2023; and promoted to Grade 7 in March 2025.
    
    \vskip 0.1cm
    }

\emptySeparator
   \researchexperience
    {Nov. 2019}   {Met Office | Modelling Infrastructure Support Systems Team{\normalfont  ~ [\href{https://www.metoffice.gov.uk/research/weather/weather-science-it/modelling-support}{\small{\websiteSymbol}}]
    }}{Exeter, UK}{}
    {Oct. 2017} {\textit{Scientific Software Engineer}\\
Developing, maintaining and supporting users of the workflow engine \href{https://github.com/cylc}{\texttt{cylc}} and infrastructure systems \href{https://github.com/metomi/rose}{\texttt{rose}} and \href{https://github.com/metomi/fcm}{\texttt{fcm}} used to configure and run scientific software for both operational forecasting and research.
    \begin{itemize}
        \itemsep0em 
        \item For example for \texttt{cylc}, implemented suite host selection and asymmetric cryptography; wrote a syntax lexer and software-checking utility; converted the full documentation from raw \LaTeX \hspace{0.1em} to Sphinx.
        \item For instance for \texttt{rose}, overhauled web applications, including porting a Python 2 utility built upon the \texttt{cherrypy} web framework to Python 3 and the \texttt{tornado} framework.
        \item In general, made numerous improvements and bug fixes; registered numerous bug reports, feature ideas and user requests; and co-delivered internal training courses and an internal update seminar.
    \end{itemize}
    }
\emptySeparator
   \researchexperience
    {March 2017} {Coltraco Ultrasonics{\normalfont  ~ [\href{https://coltraco.com/}{\small{\websiteSymbol}}]
    }}{Durham, UK}{}
    {Dec. 2015} {\textit{Student Research Assistant}\\
    Conducting scientific and technical research relating to ultrasonic measurement on a remote, part-time basis, assimilating into reports contributing to research and development of new instrumentation.
    }
\end{experiences}
\vspace{-4mm}
