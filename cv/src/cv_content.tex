% Important: this CV must be compiled with compiler that recognises academicons
% e.g. XeLaTeX.

\documentclass[10pt,a4paper]{cv_template}

\geometry{left=1cm,right=1cm,marginparwidth=6.8cm,marginparsep=1.2cm,top=1cm,bottom=1cm}

\usepackage[utf8]{inputenc}
\usepackage[T1]{fontenc}
\usepackage[default]{lato}

\usepackage{hyperref}

% Define colours to use
\definecolor{Bright}{HTML}{621E2B}
\definecolor{Green}{HTML}{7185A9}
\definecolor{Black}{HTML}{0F1F38}
\definecolor{LightGrey}{HTML}{515C50}
\definecolor{TempCol}{HTML}{B6B5AD}
\definecolor{LinkCol}{HTML}{d2e4f9}
\definecolor{DarkerBlue}{HTML}{435370}
\colorlet{heading}{Green}
\colorlet{headingdarker}{DarkerBlue}
\colorlet{accent}{Bright}
\colorlet{emphasis}{Black}
\colorlet{body}{Black}
\colorlet{linkscol}{LinkCol}

\hypersetup{
  colorlinks=false,
  urlbordercolor=linkscol
}

\begin{document}

\name{Sadie L. Bartholomew}
\tagline{\Large Computational scientist {\color{TempCol}|} Open-source contributor}
\made{\today}

\logo{5.0cm}{"cv_photo"}

\docinfo{
  {\large{\color{heading}{\Large\faGithub} \enspace GitHub:} \enspace \href{https://github.com/sadielbartholomew}{\textit{sadielbartholomew}}} \newline
  {\large{\color{heading}{\Large\aiResearchGateSquare} \enspace ResearchGate:} \enspace \href{https://www.researchgate.net/profile/Sadie_Bartholomew}{\textit{Sadie\_Bartholomew}}} \newline
  {\large{\color{heading}{\Large\aiOrcid} \enspace ORCID iD:} \enspace \href{https://orcid.org/0000-0002-6180-3603}{0000-0002-6180-3603}} \newline
  % Make sure re-add email and phone number for private CV!
  {\large{\color{heading}{\Large\faEnvelope}\enspace} sadielbartholomew@gmail.com} \newline
  {\large{\color{heading}{\Large\faPhone}\enspace} \color{TempCol}<redacted for public CV>} \newline
  {\large{\color{heading}{\Large\faMapMarker}\enspace} \enspace Reading, Berkshire, UK}
}

\makeheader

% Note these config items must go before the 'begin paracol' call
% so it knows to adjust to them:
\columnratio{0.62}
\setlength{\columnsep}{1cm}

\begin{paracol}{2}


\need{Professional experience in computing}

\steptwo{}{Computational Scientist}{11}{Jan. 2020 - present}{\href{https://ncas.ac.uk/}{National Centre for Atmospheric Science}, \href{http://cms.ncas.ac.uk/}{Computational Modelling Services group}, based at \href{https://www.reading.ac.uk/met/}{University of Reading Dept. of Meteorology}}{}

\textbf{Core work, towards two strands of Horizon 2020 project \href{https://is.enes.org/}{IS-ENES3}:}
\begin{itemize}
\itemsep0em
  \item Work relating to the \href{https://cfconventions.org/}{\textit{CF Conventions}} standard: developing, optimising and providing user services for \href{https://github.com/NCAS-CMS/cf-python}{\texttt{cf-python}, an earth science data analysis library}, and \href{https://hps.vi4io.org/_media/events/2020/summer-school-cfnetcdf.pdf}{supporting packages}, plus general support, e.g. as a member of the \href{https://cfconventions.org/governance.html}{`Information Management and Support' team}.
  \item Engineering infrastructure and supporting workflows for \href{https://github.com/ES-DOC/}{\textit{ES-DOC} (the Earth System Documentation project)}, e.g. with responsibility for the \href{https://es-doc.org/cmip6-machine-and-performance/}{\textit{CMIP6} simulation machine and performance end-to-end workflow}.
\end{itemize}

\textbf{Other and wider work undertaken as part of, and/or aligned with, role:}

\begin{itemize}
\itemsep0em
  \item Community-building as Knowledge Exchange Coordinator for the \href{https://excalibur.ac.uk/}{ExCALIBUR research programme} project \href{https://excalibur.ac.uk/projects/excalidata/}{ExCALIData}.
  \item Received a Fellowship from \href{https://www.software.ac.uk/about}{Software Sustainability Institute} (\href{https://www.software.ac.uk/about/fellows}{2022 cohort}), providing funding towards role-aligned training events.
  \item Educating (qualified as a \href{https://carpentries.org/instructors/}{Certified Instructor} for \href{https://carpentries.org/about/}{The Carpentries}), e.g:
  \begin{itemize}
  \itemsep0em
    \item Planned and delivered sessions for the \textit{ESiWACE Summer School on Effective HPC for Climate and Weather}, held August \href{https://hps.vi4io.org/events/2020/esiwace-school}{2020} and \href{https://hps.vi4io.org/events/2021/esiwace-school}{2021}.
    \item Co-delivery and resource development for CF Conventions Training (\href{http://cfconventions.org/Training/2022-Training-Workshop.html}{Sept. 2022}) and the NCAS \href{https://ncas.ac.uk/study-with-us/data-analysis-tools/}{\textit{Data Analysis Tools} day-long course}.
  \end{itemize}
  \itemsep0em
  \item Independent \href{https://sadielbartholomew.github.io/cf-standard-names-linguistics/}{analysis of} CF Standard Names to further others' efforts.
  \item Recruitment activities: shortlisting, interviewing and candidate assessment, as part of a panel, for four separate NCAS-CMS vacancies.
  \item Ad-hoc work to optimise an epidemiology model in Aug. - Oct. 2020:
  \begin{itemize}
  \itemsep0em
    \item With a small team from NCAS, optimised the performance of a new epidemiology simulation, \texttt{JUNE}, as part of research into COVID-19.
    \item Collectively produced an $\mathcal{O}(100)$ latency speedup to the pre-alpha stage, \href{https://doi.org/10.1101/2020.12.15.20248246}{receiving formal acknowledgement in the release paper}.
  \end{itemize}
  \item \href{https://github.com/sadielbartholomew/sadielbartholomew/tree/master/talks#readme}{Active} \href{https://society-rse.org/community/}{Research Software Engineering} community member.
\end{itemize}

\divider

\steptwo{}{Scientific Software Engineer}{7.5}{Oct. 2017 - Nov. 2019}{\href{https://www.metoffice.gov.uk/}{Met Office}, \href{https://www.metoffice.gov.uk/research/weather/weather-science-it/modelling-support}{Modelling Infrastructure Support Systems team}}
\textbf{Developing, maintaining and supporting users of the infrastructure systems \href{https://github.com/cylc}{\texttt{cylc}}, \href{https://github.com/metomi/rose}{\texttt{rose}} and \href{https://github.com/metomi/fcm}{\texttt{fcm}} used to configure and run scientific software for both operational forecasting and research}. For example:
\begin{itemize}
  \itemsep0em
  \item implemented suite host selection and asymmetric cryptography in, and wrote a syntax lexer and software-checking utility for, \texttt{cylc};
  \item converted the entire \texttt{cylc} documentation from raw \LaTeX \hspace{0.1em} to Sphinx;
  \item overhauled web applications, including porting a Python 2 \texttt{rose} utility utilising the \texttt{cherrypy} web framework to Python 3 and \texttt{tornado};
  \item registered numerous bug reports, bug fixes, user requests, and ideas;
  \item co-delivered internal training courses and an internal update seminar.
\end{itemize}


\need{Further Open-Source Contributions...}

...to those made as part of core work for employed roles. Only notable items are listed; for a full record see the GitHub user \textit{sadielbartholomew}.

\vspace{1em}

\stepthree{}{Reviewer for Journal of Open Source Software}{2}{June 2020+}{}{}
\begin{itemize}
  \item \textbf{Selected as reviewer for a number of submissions to the \href{https://joss.theoj.org/}{journal JOSS}}, providing feedback to see these through to publication.
\end{itemize}

\divider

\stepthree{}{Annual completion of \textit{Hacktoberfest} challenge}{2.5}{Oct. 2018+}{}{}
\begin{itemize}
  \item \textbf{Winner of prizes for three consecutive years} for completing the \href{https://hacktoberfest.digitalocean.com/}{DigitalOcean open-source initiative  \textit{Hacktoberfest}} each October.
\end{itemize}

\divider

\stepthree{}{Open personal project: creativity with \texttt{matplotlib}}{1.5}{2016+}{}{}
\begin{itemize}
  \item \textbf{Evolving an independent project, \href{https://github.com/sadielbartholomew/creative-matplotlib}{\texttt{creative-matplotlib}}}, showcasing expressive application of the Python visualisation library \texttt{matplotlib}.
\end{itemize}


\need{Publications in refereed journals}

All published, representing contributions to three different journals:

\begin{itemize}
  \item \href{https://openresearchsoftware.metajnl.com/article/10.5334/jors.359/}{V. Sochat \textit{et al.}, 2022, \textit{`The Research Software Encyclopedia: A Community Framework to Define Research Software'}, \textbf{Journal of Open Research Software} [10(1), p.2] \\({\color{Bright}\textbf{DOI: 10.5334/jors.359}})}
  \item \href{https://joss.theoj.org/papers/10.21105/joss.02717}{Hassell, D. and Bartholomew, S. L., 2020, \textit{`\texttt{cfdm}: A Python reference implementation of the CF data model'}, \textbf{Journal of Open Source Software} [5(54), p.2717] \\({\color{Bright}\textbf{DOI: 10.21105/joss.02717}})}
  \item \href{https://ieeexplore.ieee.org/document/8675433}{H. Oliver \textit{et al.}, 2019, \textit{`Workflow Automation for Cycling Systems'}, published in \textbf{Computing in Science \& Engineering} [21(4), pp.7-21] \\({\color{Bright}\textbf{DOI: 10.1109/MCSE.2019.2906593}})}
\end{itemize}


\need{Other notable work experience}

\steptwo{Student}{Research Assistant}{8}{Dec. 2015 - March 2017}{\href{https://coltraco.com/}{Coltraco Ultrasonics Ltd}}
\begin{itemize}
  \item \textbf{Conducting scientific and technical research} relating to ultrasonic measurement on a remote, part-time basis, assimilating into reports.
\end{itemize}
\divider

\steptwo{}{Science and Technology Editor}{5}{2014 - 2015; 2013 - 2014}{\href{https://www.palatinate.org.uk/?s=sadie+bartholomew}{Palatinate Newspaper};
\href{https://www.thebubble.org.uk/?s=sadie+bartholomew}{The Bubble Magazine}}
\begin{itemize}
  \item \textbf{Sourcing and writing articles to deadlines} for two separate student publications, receiving the \textit{Hunter Davies Prize for Journalism} in 2014.
\end{itemize}


\need{Education}

\steptwo{MPhys (Hons)}{Integrated Masters in Physics}{0.5}{Oct. 2012 - June 2017}{Durham University}
\begin{itemize}
  \item \textbf{Final-year computational project} entitled \textit{`Searches for
boosted top quarks'}, evaluating and refining top-tagging algorithms implemented in C++, as applied to simulated LHC proton-proton scattering events.
  \item \textbf{\href{https://www.dur.ac.uk/physics/modules/}{Undergraduate physics syllabus plus elective masters-level modules}} on Advanced Quantum Theory, Particle Theory and Cosmology.
\end{itemize}
\divider

\steptwo{}{Secondary Education}{14.5}{2007 - 2012}{Ponteland Community High School}
\begin{itemize}
  \item \textbf{A levels}: Physics (A*), Chemistry (A*), Mathematics (A*), Further Mathematics (A), Critical Thinking (A)
  \item \textbf{Other}: \href{https://qips.ucas.com/qip/extended-project-qualification-epq}{Extended Project Qualification} (A*), \href{https://www.crestawards.org/crest-gold}{CREST Gold Award}
  \item \textbf{GCSE}: 11 A* qualifications including French and self-taught History
\end{itemize}

\switchcolumn


\need{Technical Skills}

\textbf{Key}: \LEFTcircle \enspace ability self-assessment | $\bigstar$ preferred

\divider

\makesubtitle{PROGRAMMING LANGUAGES:}
\vspace{0.5ex}

\begin{itemize}
  \item Professional and/or academic use of:
  \vspace{0.5ex}
    \itemtag{Python} \grade{4.5}(see also below $\boldsymbol{\rightarrow}$) \newline
    \itemtag{Unix shell} \Big(\itemtag{Bash $\bigstar$}\Big) \grade{4.5} \newline
    \itemtag{JavaScript} \grade{3.5} \newline
    \itemtag{C++} \grade{3} \newline
    \itemtag{Emacs Lisp} \grade{2.5} \newline
  \item Recreational use of, as well as the above:
  \vspace{0.5ex}
    \itemtag{Common Lisp} \grade{2} \newline
    \itemtag{Haskell} \grade{1.5} \newline
\end{itemize}

\makesubtitle{$\boldsymbol{\rightarrow}$ notable Python ecosystem competence:}
\vspace{0.5ex}

\itemtag{NumPy} \itemtag{SciPy} \itemtag{Dask} \itemtag{\texttt{matplotlib}}
\itemtag{Jupyter} \itemtag{Sphinx} \itemtag{\texttt{mpi4py}} \itemtag{Numba}
\itemtag{Airflow} \itemtag{Tornado} \itemtag{\texttt{netcdf4-python}}

\divider

\makesubtitle{MARKUP LANGUAGES:}
\vspace{0.5ex}

\begin{itemize}
  \item Intermediate to advanced ability with: \newline
    \itemtag{Markdown}
    \itemtag{reStructuredText}
    \itemtag{HTML}
    \itemtag{YAML}
    \itemtag{\LaTeX}
    \itemtag{INI}
  \item Basic proficiency with: \newline
    \itemtag{XML}
    \itemtag{TOML}
    \itemtag{Emacs Org-mode}
\end{itemize}

\divider

\makesubtitle{INFRASTRUCTURE AND SYSTEMS:}
\vspace{1.5ex}

\begin{itemize}
  \item OS: \itemtag{Linux $\bigstar$} \itemtag{Windows}
  \item Version control: \itemtag{git $\bigstar$} \itemtag{SVN}
  \item Development environment: \itemtag{emacs $\bigstar$}
  \item Scheduling: \itemtag{slurm} \itemtag{PBS} \itemtag{cron} \itemtag{at}
  \item Host: \itemtag{GitHub $\bigstar$} \itemtag{GitLab} \itemtag{Bitbucket}
  \item Web technologies: \itemtag{CSS} \itemtag{Vue.js}
  \item Other notable: \itemtag{conda} \itemtag{MPI} \itemtag{SQL}
\end{itemize}


\need{Soft Skills}

Five soft skills to highlight:
\vspace{0.5ex}

\itemtag{International collaboration}
\itemtag{Curiosity}
\itemtag{Continuous learning}
\itemtag{Creativity}
\itemtag{Presenting in-person \& virtually (see below)}

\need{Selected presentations}

(All delivered talks listed \href{https://github.com/sadielbartholomew/sadielbartholomew/tree/master/talks#readme}{on this webpage}.)
\begin{itemize}
  \item \href{https://github.com/sadielbartholomew/sadielbartholomew/blob/master/talks/is-enes3-ga2-es-doc-status.pdf}{\textit{`Status of ES-DOC'}}, presented (individually) virtually at the \textbf{IS-ENES3 Second General Assembly} in October 2021.
  \item \href{https://metomi.github.io/presentations/RSEConUK2019-Cylc-Talk/#/title-slide}{\textit{`Pursuing and supporting reproducible workflows for all with Cylc'}}, \href{https://rseconuk2019.sched.com/event/QSSI/5d2-hpc-pursuing-and-supporting-reproducible-workflows-for-all-with-cylc}{co-presented in-person at the \textbf{Fourth Conference of Research Software Engineering} ("RSEConUK 2019") in September 2019}.
\end{itemize}


\need{Engineering experience}

Practiced in software engineering processes:
\vspace{-1.0ex}
% Why does the above value need to be different to the soft skills
% section to give same gap (note it has nothing to do with this comment)

\itemtag{Software development life cycle stages}
\itemtag{Use of Tier-1, -2 and -3 HPC facilities}
\itemtag{Design patterns}
\itemtag{User support}
\itemtag{Peer review of code and written proposals}
\itemtag{Documentation writing and infrastructure}
\itemtag{Continuous integration}
\itemtag{DevOps}


\need{Professional Interests}

\itemtag{Infrastructure relating to (earth) simulation}
\itemtag{Metadata}
\itemtag{Workflows}
\itemtag{Open source}
\itemtag{High-performance computing}
\itemtag{Big data}
\itemtag{Data models}
\itemtag{Performant Python}
\itemtag{UX}


\need{Other Interests}

\itemtag{Art}
\itemtag{Guitar}
\itemtag{Rock music}
\itemtag{Chess}
\itemtag{Nature}
\itemtag{Walking}
\itemtag{Reading}
\itemtag{Cookery}

\need{References}

Available on request.


\need{R\'{e}sum\'{e} metadata}

\begin{itemize}
  \item Last update: \today
\end{itemize}

\end{paracol}

\end{document}
